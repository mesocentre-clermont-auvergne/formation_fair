\section{Conda ecosystem}
\subsection{a case of bioconda}

\begin{frame}{Conda system}
\begin{itemize}
\item \hyperlink{https://www.anaconda.com/}{Anaconda}
	\begin{itemize}
	\item Open source distribution
    \item Cross platform
    \item Available on cluster without admin whrite
    \item Thousands of available tool in informatic and bioinformatic
	\end{itemize}
\item \hyperlink{https://docs.conda.io/en/latest/miniconda.html}{Miniconda}
	\begin{itemize}
	\item A lightheight Anaconda version with minimal requirment
	\item Same advantages ad Anaconda
	\end{itemize}
\item \hyperlink{https://docs.conda.io/projects/conda/en/latest/index.html}{Conda}
	\begin{itemize}
	\item Package manager AND environment manager
	\item installed with Ana pr Miniconda
	\item Python based but can also install tools from R, C++ or Julia...
	\end{itemize}
\end{itemize}
\includegraphics[width=0.1\textwidth]{images/conda_sheet_4.12.pdf} 
\includegraphics[width=0.1\textwidth]{images/conda_logo.pdf} 
\end{frame}

\begin{frame}{The channels and the tools}
The tools are packaged and available on several "channels"
\begin{itemize}
\item conda-forge
\item anaconda
\item R
\item Bioconda \footnote{Bioconda: sustainable and comprehensive software distribution for the life sciences \textit{Grüning et al.}, Nature methods, 2018. DOI 10.1038/s41592-018-0046-7} --> Most of the bioinformatic tools
\end{itemize}
\includegraphics[width=0.1\textwidth]{images/conda_env_detail.pdf} 
\end{frame}

\begin{frame}
presenter les commandes de base de conda pour lister des env lister des packages 
presenter ke fait de gerer les version et la compatibilite des versions
presenter la rsolution des enritonnements et la lenteur de conda donc aller vers mamba

\end{frame}

\begin{frame}
\includegraphics[width=0.1\textwidth]{images/reproductibility.jpg} \footnote{Practical Computational Reproducibility in the Life Sciences
Grüning et al, Cell Systems, 2018. DOI 10.1016/j.cels.2018.03.014}

\end{frame}